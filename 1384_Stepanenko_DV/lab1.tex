\documentclass{article}
\usepackage{ucs} 
\usepackage[utf8x]{inputenc} % Включаем поддержку UTF8  
\usepackage[russian,english]{babel}
\usepackage{pgfplots}
\usepackage{amsfonts} 
\pgfplotsset{compat=1.9}

\begin{document}
\selectlanguage{russian}
\textbf{Степаненко Денис (1384)}

\begin{center}
Лабораторная работа №1 "Исследование двумерного нормального распределения"\\
\end{center}

\textbf{Цель работы}\\
Исследовать свойства двумерного случайного вектора имеющего нормальное распределение, овладеть навыками преобразования нормального вектора в стандартный и в вектор с независимыми компонентами.\\


\textbf{Выполнение работы}\\
Плотность двумерного нормального разпределениея имеет вид:
$\rho_{\xi,\eta}(x,y)=C\cdot(exp(-\frac{1}{2}(3x^2+3xy+7y^2-9x+8y+13)))$\\

1. Вычисление числовых характеристик СВ и построение графига его плотности распределения.\\
 
$3x^2+3xy+7y^2-9x+8y+13 = 3(x^2+xy-3x)+7y^2+8y+13
=3(x^2+2x(0.5y-1.5)+(0.5y-1.5)^2-(0.5y-1.5)^2)+7y^2+8y+13
=3(x+0.5y-1.5)^2+7y^2+8y+13-\frac{3y^2}{4}+\frac{9y}{2}+\frac{27}{4}
=3(x+0.5y-1.5)^2+\frac{25y^2+(32+18)y+(13\cdot14-27)}{4}
=3(x+0.5y-1.5)^2+\frac{25y^2+50y+25}{4}
=3(x+0.5y-1.5)^2+\frac{25}{4}(y+1)^2
=3(x+\frac{1}{2}(y+1)-2)^2+\frac{25}{4}(y+1)^2
=3[(x-2)^2+\frac{2}{2}(y+1)(x-2)+\frac{1}{4}(y+1)^2]+\frac{25}{4}(y+1)^2
=3(x-2)^2+3(x-2)(y+1)+7(y+1)$\\

$\mathbb{E}_{\xi,\eta}=\left({\begin{array}{c}
    2\\
    -1\\
  \end{array} } \right)$\\
  
$\Sigma_{\xi,\eta}^{-1}=\left({\begin{array}{cc}
    3 & 1.5\\
    1,5 & 7\\
  \end{array} } \right)$\\

$\Sigma_{\xi,\eta}=\frac{4}{75}\left({\begin{array}{cc}
    7 & -\frac{3}{2}\\
     -\frac{3}{2} & 3\\
  \end{array} } \right)$\\

$\mathbb{D}_\xi=7$\\

$\mathbb{D}_\eta=3$\\

$\mathop{\mathrm{cov}}\nolimits(\xi, \eta)=-1.5$\\

$\rho_{\xi,\eta}=\frac{\mathop{\mathrm{cov}}\nolimits(\xi, \eta)}{\sqrt[•]{\mathbb{D}_\xi\cdot\mathbb{D}_\eta}}=\frac{-1.5}{\sqrt[]{3\cdot7}}=-\frac{3}{2\sqrt[]{21}}$\\

$C=\frac{1}{(2\pi)^{\left(\frac{n}{2}\right)}\sqrt[•]{\det\Sigma}}=\frac{1}{2\pi\sqrt[•]{21+2.25}}$\\

$p_{\xi,\eta}(x,y)=\frac{1}{2\pi\sqrt[•]{23.25}}exp\left(
-\frac{1}{2}
\left[
  	\left(
  		{\begin{array}{c}
    			x\\
    			y\\
  		\end{array}}
  	\right)
-
  	\left(
  		{\begin{array}{c}
    			2\\
    			-1\\
  		\end{array}}
  	\right)
\right]^\top
\left(
	{\begin{array}{cc}
  		3 & 1.5\\
  		1,5 & 7\\
	\end{array}}
\right)
\left[
 	 \left(
  		{\begin{array}{c}
    			x\\
    			y\\
  		\end{array}}
  	\right)
-
  	\left(
  		{\begin{array}{c}
    			2\\
    			-1\\
  		\end{array}}
  	\right)
\right]
\right)$\\


График плотности распределения случайного вектора:

\begin{center}
\begin{tikzpicture}
\begin{axis}[ 	
	view={45}{20},
	colormap/greenyellow,
	colorbar
]
\addplot3[	
	domain=0:4,
	domain y = -2.3:0.3,
	samples = 50,
	samples y = 50,
	surf] {1/sqrt(2*pi*23.25)*e^(-1/2*(3*x^2+3*x*y+7*y^2-9*x+8*y+13)};
\end{axis}
\end{tikzpicture}
\end{center}

\newpage
2. Нахождение афинного преобразования, приводящее исходный СВ в стандартный нормальный.\\

Афинное преобразование - математическая операция, которая сопоставляет одно координатное пространство с другим. Формула афинного преобразования:
$$ \vec{\xi_{st}}=A\vec{\xi}+B$$

$3x^2+3xy+7y^2-9x+8y+13 = 3(x^2+xy-3x)+7y^2+8y+13
=3(x^2+2x(0.5y-1.5)+(0.5y-1.5)^2-(0.5y-1.5)^2)+7y^2+8y+13
=3(x+0.5y-1.5)^2+7y^2+8y+13-\frac{3y^2}{4}+\frac{9y}{2}+\frac{27}{4}
=3(x+0.5y-1.5)^2+\frac{25y^2+(32+18)y+(13\cdot14-27)}{4}
=3(x+0.5y-1.5)^2+\frac{25y^2+50y+25}{4}
=3(x+0.5y-1.5)^2+\frac{25}{4}(y+1)^2=(\sqrt[•]{3}x+\frac{\sqrt[•]{3}}{2}y-\frac{3\sqrt[•]{3}}{2})^2+(\frac{5}{2}y+\frac{5}{2})^2$\\

Таким образом, получается:
$ 	\left(
  		{\begin{array}{c}
    			\xi_{st}\\
    			\eta_{st}\\
  		\end{array}}
  	\right)
  	=
  	\left(
  		{\begin{array}{cc}
    			\sqrt[•]{3} & \frac{\sqrt[•]{3}}{2}\\
    			0 & \frac{5}{2} \\
  		\end{array}}
  	\right)
  	\left(
  		{\begin{array}{c}
    			\xi\\
    			\eta\\
  		\end{array}}
  	\right)
  	+
  	\left(
  		{\begin{array}{c}
    			-\frac{3\sqrt[•]{3}}{2}\\
    			\frac{5}{2}\\
  		\end{array}}
  	\right)
$\\

Чтобы показать, что вектор нормальный, необходимо убедиться в равенствах:

$$\mathbb{E}_{\xi_{st},\eta_{st}}=
\left({
	\begin{array}{c}
  		0\\
  		0\\
  	\end{array} } 
\right)$$

$$\Sigma_{\xi_{st},\eta_{st}}=
\left(
	{\begin{array}{cc}
    1 & 0\\
    0 & 1\\
  \end{array}}
\right)$$

Итак, поехали:\\

$\mathbb{E}_{\xi_{st},\eta_{st}}=
\left(
    {\begin{array}{cc}
    	\sqrt[•]{3} & \frac{\sqrt[•]{3}}{2}\\
    		0 & \frac{5}{2} \\
  	\end{array}}
\right)
\left(
	{\begin{array}{c}
  		2\\
  		-1\\
  	\end{array}} 
\right)
+
\left(
  	{\begin{array}{c}
    		-\frac{3\sqrt[•]{3}}{2}\\
    		\frac{5}{2}\\
  	\end{array}}
\right)
=
\left(
	{\begin{array}{c}
  		0\\
  		0\\
  	\end{array}} 
\right)$\\

$\Sigma_{\xi_{st},\eta_{st}}=\frac{4}{75\cdot4}
  	\left(
  		{\begin{array}{cc}
    			2\sqrt[•]{3} & \sqrt[•]{3}\\
    			0 & 5 \\
  		\end{array}}
  	\right)
  	\left(
		{\begin{array}{cc}
    			7 & -\frac{3}{2}\\
     		-\frac{3}{2} & 3\\
  		\end{array}}
  	\right)
  	\left(
  		{\begin{array}{cc}
    			2\sqrt[•]{3} & 0\\
    			\sqrt[•]{3} & 5 \\
  		\end{array}}
  	\right)
  	=
  	\left(
  		{\begin{array}{cc}
    			\frac{\sqrt[•]{3}}{2} & 0\\
    			\frac{15}{2} & 15 \\
  		\end{array}}
  	\right)
  	\left(
  		{\begin{array}{cc}
    			2\sqrt[•]{3} & 0\\
    			\sqrt[•]{3} & 5 \\
  		\end{array}}
  	\right)
  	= \frac{1}{75}
  	\left(
  		{\begin{array}{cc}
    			75 & 0\\
    			0 & 75 \\
  		\end{array}}
  	\right)=
  	\left(
  		{\begin{array}{cc}
    			1 & 0\\
    			0 & 1\\
  		\end{array}}
  	\right)
$\\

Таким образом, полученный вектор через афинно преоразование является стандарнтым нормальным СВ. \newpage

3. Нахождение ортогонального преобразования, приводящего соответствующий центрированный СВ в вектор с независимыми компонентами. Построение графика плотности полученного распределения.\\

Найдем определитель матрицы коварианций и прировняем его к 0.\\

$\det(\Sigma^{-1}-\lambda E)=
\det
\left(
  	{\begin{array}{cc}
    		3-\lambda & 1.5\\
    		1.5 & 7-\lambda\\
  	\end{array}}
\right)
= (3-\lambda)\cdot(7-\lambda)-1.5^2=\lambda^2-10\lambda+18.75=0$\\

Решая квадратное уравнение получаем собственные значения:
$\lambda_{1}=\frac{15}{2}$
$\lambda_{2}=\frac{5}{2}$\\

Теперь найдем собственные векторы:\\

1)Для собственного ветора $\lambda_{1}=7.5$:\\

$$\left(
  	{\begin{array}{cc}
    		-4.5 & 1.5\\
    		1.5 & -0.5\\
  	\end{array}}
\right)
\left(
  	{\begin{array}{c}
    		v_{1x}\\
    		v_{1y}\\
  	\end{array}}
\right)=
\left(
  	{\begin{array}{c}
    		0\\
    		0\\
  	\end{array}}
\right)
$$
Отсюда следует отношение: $v_{1y}=3v_{1x}$.\\
Выберем значения собственного вектора:
$v_{1}=\left(
  	{\begin{array}{c}
    		1\\
    		3\\
  	\end{array}}
\right)$\\

2)Для собственного ветора $\lambda_{2}=2.5$:\\

$$\left(
  	{\begin{array}{cc}
    		0.5 & 1.5\\
    		1.5 & 4.5\\
  	\end{array}}
\right)
\left(
  	{\begin{array}{c}
    		v_{2x}\\
    		v_{2y}\\
  	\end{array}}
\right)=
\left(
  	{\begin{array}{c}
    		0\\
    		0\\
  	\end{array}}
\right)
$$
Отсюда следует, что: $v_{2x}=-3v_{2y}$.\\
Выберем значения собственного вектора:
$v_{2}=\left(
  	{\begin{array}{c}
    		3\\
    		-1\\
  	\end{array}}
\right)$\\

Рассчитаем норму векторов: $\|v_{1}\|=\sqrt[•]{1^{2}+3^{2}}=\sqrt[•]{10}=\|v_{2}\|$\\

Далее составим матрицу из собственных линейно независимыx векторов и величины, обратной норме:\\
$$Q^\top=\frac{1}{\sqrt[•]{10}}
\left(
  	{\begin{array}{cc}
    		1 & 3\\
    		3 & -1\\
  	\end{array}}
\right)$$
$$Q=\frac{1}{\sqrt[•]{10}}
\left(
  	{\begin{array}{cc}
    		1 & 3\\
    		3 & -1\\
  	\end{array}}
\right)$$

Найдем значение СВ в ортогональной системе координат:
$$\left(
  	{\begin{array}{c}
    		\xi_{ort}\\
    		\eta_{ort}\\
  	\end{array}}
\right)=Q
\left(
  	{\begin{array}{c}
    		\xi\\
    		\eta\\
  	\end{array}}
\right)=
\left(
  	{\begin{array}{c}
    		\xi+3\eta\\
    		3\eta-\xi\\
  	\end{array}}
\right)$$

А также некоторые числовые характеристики:

1)Учитывая, что 
$\mu_{ист}=
\left(
    {\begin{array}{c}
    		2\\
    		1\\
  	\end{array}}
\right)
$
найдем математическое ожидание СВ в ортогональной СК путем домножение на матрицу $Q$ слева:
$$\mathbb{E}_{\xi_{ort},\eta_{ort}}=Q\mu=
\left(
    {\begin{array}{cc}
    		1 & 3\\
    		3 & -1\\
  	\end{array}}
\right)
\left(
    {\begin{array}{c}
    		2\\
    		1\\
  	\end{array}}
\right)=
\left(
    {\begin{array}{c}
    		5\\
    		5\\
  	\end{array}}
\right)$$

2)Маттрица ковариаций:\\
$\Sigma_{ort}=Q\Sigma Q^\top=
\frac{4}{750}
\left(
  	{\begin{array}{cc}
    		1 & 3\\
    		3 & -1\\
  	\end{array}}
\right)
\left(
	{\begin{array}{cc}
    		7 & -\frac{3}{2}\\
     	-\frac{3}{2} & 3\\
  	\end{array}} 
\right)
\left(
  	{\begin{array}{cc}
    		1 & 3\\
    		3 & -1\\
  	\end{array}}
\right)=\frac{4}{750}
\left(
  	{\begin{array}{cc}
    		2.5 & 7.5\\
    		22.5 & -7.5\\
  	\end{array}}
\right)
\left(
  	{\begin{array}{cc}
    		1 & 3\\
    		3 & -1\\
  	\end{array}}
\right)=\frac{4}{750}
\left(
  	{\begin{array}{cc}
    		25 & 0\\
    		0 & 75\\
  	\end{array}}
\right)=
\left(
  	{\begin{array}{cc}
    		\frac{2}{15} & 0\\
    		0 & \frac{2}{5}\\
  	\end{array}}
\right)
$\\


3) Обратная (для построения уравнения плотности распределения):\\
$\Sigma_{ort}^{-1}=
\left(
    {\begin{array}{cc}
    		\lambda_{1} & 0\\
    		0 & \lambda_{2}\\
  	\end{array}}
\right)=
\left(
    {\begin{array}{cc}
    		7.5 & 0\\
    		0 & 2.5\\
  	\end{array}}
\right)$\\

Составим уравнение плотности распределения, используя полученные величины:

$p_{\xi,\eta}=\frac{\sqrt[•]{750}}{4\pi}exp\left(
-\frac{1}{2}
\left[
  	\left(
  		{\begin{array}{c}
    			x\\
    			y\\
  		\end{array}}
  	\right)
-
  	\left(
  		{\begin{array}{c}
    			5\\
    			5\\
  		\end{array}}
  	\right)
\right]^\top
\left(
	{\begin{array}{cc}
  		7.5 & 0\\
  		0 & 2.5\\
	\end{array}}
\right)
\left[
 	 \left(
  		{\begin{array}{c}
    			x\\
    			y\\
  		\end{array}}
  	\right)
-
  	\left(
  		{\begin{array}{c}
    			5\\
    			5\\
  		\end{array}}
  	\right)
\right]
\right)
=
\frac{\sqrt[•]{750}}{4\pi}exp\left(
-\frac{1}{2}
\left(
	{\begin{array}{cc}
  		7.5(x-5) & 2.5(y-5)\\
	\end{array}}
\right)
\left(
  	{\begin{array}{c}
    		x-5\\
    		y-5\\
  	\end{array}}
\right)
\right)
=
\frac{\sqrt[•]{750}}{4\pi}exp\left(-\frac{1}{2}(7.5x^2-75x+2.5y^2-25y+250)\right)$\\

График плотности распредления:\\

\begin{tikzpicture}
\begin{axis}[ 	
	view={45}{20},
	colormap/cool,
	colorbar
]
\addplot3[	
	domain=3.5:6.5,
	domain y = 2.5:7.5,
	samples = 50,
	samples y = 50,
	surf] {sqrt(750)/(4*pi)*e^(-1/2*(7.5*x^2+2.5*y^2-75*x-25*y+250)};
\end{axis}
\end{tikzpicture}

\newpage

4. Вычисление характеристик распределения случайного вектора $(-5\xi+5\eta, -5\xi+4\eta)^\top$. Нахождение формулы и построение графика плотности распределения полученного СВ.\\

$\left(
  	{\begin{array}{c}
    		\xi_{4}\\
    		\eta_{4}\\
  	\end{array}}
\right)
=
\left(
  	{\begin{array}{c}
    		-5\xi+4\eta\\
    		-5\xi+4\eta\\
  	\end{array}}
\right)
$\\

$
\left(
  	{\begin{array}{cc}
    	-5 & 4\\
    	-5 & 4\\
  	\end{array}}
\right)
\cdot
\left(
  	{\begin{array}{c}
    		\xi\\
    		\eta\\
  	\end{array}}
\right)
$

$\det A=0$, заменим ее на другую матрицу:
$\left(
  	{\begin{array}{c}
    		\xi_{4}\\
    		\eta_{4}\\
  	\end{array}}
\right)
=
\left(
  	{\begin{array}{c}
    		-5\xi-4\eta\\
    		-5\xi+4\eta\\
  	\end{array}}
\right)
$

Продолжим:\\

$\mathbb{E}_{\xi_{4},\eta_{4}}=A\cdot \mathbb{E}_{\xi,\eta}=
\left(
  	{\begin{array}{cc}
    	-5 & -4\\
    	-5 & 4\\
  	\end{array}}
\right)
\left(
  	{\begin{array}{c}
    		2\\
    		-1\\
  	\end{array}}
\right)=
\left(
  	{\begin{array}{c}
    		-6\\
    		-14\\
  	\end{array}}
\right)$\\

$\Sigma_{\xi_{4},\eta_{4}}=A\cdot\Sigma\cdot A^\top=\frac{4}{75}
\left(
	{\begin{array}{cc}
    		-5 & -4\\
     	-5 & 4\\
  	\end{array}} 
\right)
\left(
	{\begin{array}{cc}
    		7 & -\frac{3}{2}\\
     	-\frac{3}{2} & 3\\
  	\end{array}} 
\right)
\left(
	{\begin{array}{cc}
    		-5 & -5\\
     	-4 & 4\\
  	\end{array}} 
\right)=\frac{4}{75}
\left(
	{\begin{array}{cc}
    		-29 & -4.5\\
     	-41 & 19.5\\
  	\end{array}} 
\right)
\left(
	{\begin{array}{cc}
    		-5 & -5\\
     	-4 & 4\\
  	\end{array}} 
\right)=\frac{4}{75}
\left(
	{\begin{array}{cc}
    		163 & 127\\
     	127 & 283\\
  	\end{array}} 
\right)
$\\

$\Sigma_{\xi_{4},\eta_{4}}^{-1}=\frac{1}{1600}\left({\begin{array}{cc}
    283 & -127\\
    -127 & 163\\
  \end{array} } \right)$\\

$\mathbb{D}_{\xi_4}=\frac{163\cdot4}{75}=\frac{652}{75}$\\

$\mathbb{D}_{\eta_4}=\frac{283\cdot4}{75}=\frac{1132}{75}$\\

$\mathop{\mathrm{cov}}\nolimits(\xi_{4}, \eta_{4})=\frac{127\cdot4}{75}=\frac{508}{75}$\\

$\rho_{\xi_{4},\eta_{4}}=\frac{\mathop{\mathrm{cov}}\nolimits(\xi_{4}, \eta_{4})}{\sqrt[•]{\mathbb{D}_{\xi_{4}}\cdot\mathbb{D}_{\eta_{4}}}}=\frac{508}{\sqrt[]{652\cdot1132}}=\frac{508}{\sqrt[•]{738064}}$\\

$C=\frac{1}{(2\pi)^{\left(\frac{n}{2}\right)}\sqrt[•]{\det\Sigma}}=\frac{1}{2\pi\sqrt[•]{1600}}=\frac{1}{800\pi}$\\

$\rho_{\xi_{4},\eta_{4}}(x,y)=\frac{1}{800\pi}exp\left(
-\frac{1}{2}
\left[
  	\left(
  		{\begin{array}{c}
    			x\\
    			y\\
  		\end{array}}
  	\right)
-
  	\left(
  		{\begin{array}{c}
    			-6\\
    			-14\\
  		\end{array}}
  	\right)
\right]^\top
\frac{1}{1600}
\left(
	{\begin{array}{cc}
  		283 & -127\\
  		-127 & 163\\
	\end{array}}
\right)
\left[
 	 \left(
  		{\begin{array}{c}
    			x\\
    			y\\
  		\end{array}}
  	\right)
-
  	\left(
  		{\begin{array}{c}
    			-6\\
    			-14\\
  		\end{array}}
  	\right)
\right]
\right)
=
\frac{1}{800\pi}exp\left(
-\frac{1}{3200}
  	\left(
  		{\begin{array}{c}
    			283x-127y+1698-1778\\
    			163y-127x+2282-762\\
  		\end{array}}
  	\right)^\top
  	\left(
  		{\begin{array}{c}
    			x+6\\
    			y+14\\
  		\end{array}}
  	\right)
\right)
=$\\
$\frac{1}{800\pi}exp\left(-\frac{1}{3200}(283x^2-254xy-160x+163y^2+3040y+20800)\right)$\\

График плотности распределения:\\

\begin{tikzpicture}
\begin{axis}[ 	
	view={45}{20},
	colormap/violet,
	colorbar
]
\addplot3[	
	domain=-16:4,
	domain y = -24:-4,
	samples = 50,
	samples y = 50,
	surf] {1/sqrt(800*pi)*e^(-1/3200*(283*x^2-254*x*y+163*y^2
	-160*x+3050*y+20800)};
\end{axis}
\end{tikzpicture}
\newpage

5. Нахождение условного распределения $\xi$ при условии $\eta$. Вычисление математического ожидания и дисперсии.\\

Для нахождения условного распределения воспользуемся формулой:$$\rho_{\xi|\eta=y}(x)=\frac{\rho_{\xi,\eta}(x,y)}{\rho_{\eta}(y)},$$ где $\rho_{\eta}(y)$ можно найти по формуле:
$$\int_{-\infty}^\infty\rho_{\xi,\eta}(x,y)dy$$

Давайте вычислим интеграл:\\

$\int_{-\infty}^\infty \frac{1}{2\pi\sqrt[•]{23.25}}exp(-\frac{1}{2}(3x^2+3xy+7y^2-9x+8y+13))dy=\frac{1}{\sqrt[•]{2\pi\cdot7\cdot23.25}}exp(-\frac{75}{56}(x-2)^2)$\\
 
Получили распределение случайной величины $\eta$. Теперь найдем условное распределение\\

$\rho_{\xi|\eta=y}(x)= \frac{\frac{1}{2\pi\sqrt[•]{23.25}}exp(-\frac{1}{2}(3x^2+3xy+7y^2-9x+8y+13))}{\frac{1}{\sqrt[•]{2\pi\cdot7\cdot23.25}}exp(-\frac{75}{56}(x-2)^2)}=\sqrt[•]{\frac{7}{2\pi}}exp(-\frac{1}{56}(9x^2+84x+48x+196y^2+224y+64))$\\

Далее вычислим дисперсию и математисеское ожидание для полученного условного распределения:\\

$\mathbb{E}_{\xi|\eta=y}=\int_\mathbb{R}x\cdot\rho_{\xi|\eta=y}(x)dx=\int_{-\infty}^\infty x\cdot\sqrt[•]{\frac{7}{2\pi}}exp(-\frac{1}{56}(9x^2+84x+48x+196y^2+224y+64))=-\frac{44}{9}\sqrt[•]{14\pi} exp(-\frac{1}{2}(y-1)(7y+15))$\\

$\mathbb{D}_{\xi|\eta=y}=\int_\mathbb{R}x^2\cdot\rho_{\xi|\eta=y}(x)dx=\int_{-\infty}^\infty x^2\cdot\sqrt[•]{\frac{7}{2\pi}}exp(-\frac{1}{56}(9x^2+84x+48x+196y^2+224y+64))=\frac{1024}{27}\sqrt[•]{14\pi} exp(-\frac{1}{2}(y-1)(7y+15))$\\
\newpage
\textbf{Выводы}\\
В ходе выполнения лабораторной работы было изучено многомерное нормальное распределение, уравнение его плотности распределения, основные числовые характеристики, свойтсва. Также были закреплены навыки вычисления числовых характеристик многомерного СВ и построение графиков его распределения. Таким образом, мы провели афинное преобразование, нашли ортогональное преобразование, описали два новых случайных распределения, одним из которых было условное.

\end{document}